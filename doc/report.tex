\documentclass[12pt]{article}
\usepackage[T1]{fontenc}
\usepackage[utf8]{inputenc}
\usepackage[spanish]{babel}
\usepackage[letterpaper, portrait, margin=1.3in]{geometry}
\usepackage{coffee3}
\usepackage{graphicx}
\usepackage{fancyhdr}
\usepackage{color}
\usepackage{lipsum}
\usepackage{amsmath}
\usepackage{listings}
\usepackage{mfirstuc}
\usepackage{tikz}
\usepackage{eurosym}
\usepackage{marginnote}

\definecolor{pblue}{rgb}{0.13,0.13,1}
\definecolor{pgreen}{rgb}{0,0.5,0}
\definecolor{pred}{rgb}{0.9,0,0}
\definecolor{pgrey}{rgb}{0.46,0.45,0.48}

\lstset {
    commentstyle=\color{pgreen},
    showstringspaces=false,
    keywordstyle=\color{pblue},
    stringstyle=\color{pred},
    basicstyle=\footnotesize\ttfamily,
    rulecolor=\color{black}
}

\pagestyle{fancy}
\fancyhf{}
\renewcommand{\headrulewidth}{0pt}
\setlength\headheight{25pt}
\fancyhead[L]{\S\ D. Lilue}
\fancyhead[R]{\color{pgrey}\textsc{{\footnotesize \leftmark}}\ $|$ \color{black} \marginnote{\thepage}}
\fancyfoot[R]{\color{pgrey}\footnotesize Universidad Politécnica de Madrid}

\renewcommand\spanishtablename{Tabla}
\renewcommand*\spanishlisttablename{Índice de tablas}

\begin{document}

% Title
\begin{center}
    \phantom{x}
    \vfill
    {\Huge \textsc{Maximizar ganancias al asignar vuelos de una aerolínea}}
    \vfill
    \includegraphics[scale=.23]{./assets/aeroplane-2026921_960_720.png}
    \includegraphics[scale=.37]{./assets/learjet-private-jet-coloring-page.jpg}
    \vfill
    Junio, 2018
    
    \vspace{2em}
    
    {\large David Lilue}
    \vfill
    {\Large Universidad Politécnica de Madrid}

    {\Large Máster en Software y Sistemas}
    \vfill
    %\coffee{1}
\end{center}

\cofeB{0.175}{0.5}{180}

\thispagestyle{empty}

\newpage

\tableofcontents
\listoffigures
\listoftables
\thispagestyle{empty}

\newpage

\section{Introducción}

Este trabajo tiene como objetivo buscar la solución óptima a un problema que plantea condiciones y restricciones para programar los vuelos de una aerolínea. Este problema se enfoca en un grupo finito de ciudades y distintos tipos de aeronaves, estas condiciones nos brindan la materia prima para empezar a crear un modelo en el que se pueda aplicar algún método de programación lineal. En principio se habla un poco de que nos permite la programación lineal, los elementos que la conforman y los puntos de vista que puede tener. Posteriormente se enuncia el problema a resolver, así como los datos que brinda una aerolínea ficticia con el objetivo de maximizar sus ganancias.

Hay una sección dedicada al modelo planteado para abordar a este problema, la variables y constantes que lo conforman. Además de eso, se formalizan las restricciones necesarias para encerrar el espacio de búsqueda, por lo tanto el método numérico que se use podrá hallar la solución óptima. Este acordonamiento de las soluciones es el resultado de condiciones sujetas a las demandas de la aerolínea y tienen una sección para explicarlas. Finalizando, se pueden ver progresivamente los distintos resultados obtenidos en conjunto con una breve interpretación, permitiendo terminar con una conclusión del trabajo.

\newpage

\section{Programación lineal}

A través de la programación lineal es posible explorar un espacio de soluciones posibles, con el objetivo de encontrar la óptima. Existen distintos métodos como \emph{Simplex} y el del punto interior, cada una con su propio heurística pero como un mismo fin. Estos técnicas lograr atacar efectivamente problemas que existen en todo tipo de ámbito, con el objetivo de minimizar o maximizar una función. Esta optimización está sujeta un conjunto de restricciones y constantes que permiten reducir el espacio de búsqueda. Con lo cual es posible encontrar solución a problemas adaptados a escenarios de la vida real.

La función objetivo está conformada por variables de decisión, siendo esta un polinomio, la meta es encontrar aquella permutación que logre retornar el menor o mayor resultado; dependiendo de lo que se desee obtener. Por ello, el problema que se presenta en este trabajo se considera dentro del conjunto de los problemas de toma de decisión y en las secciones siguientes se describirá el problema de la programación de vuelos. Encontrando la solución con \texttt{Octave}, un lenguaje de programación \emph{open-source} para resolver problemas de computación numérica; una versión gratuita de \emph{MATLAB}.

\section{Asignación de vuelos}

Este problema tiene como objetivo maximizar las ganancias para una aerolínea, para ello es necesario tomar una decisión en la programación de los vuelos durante un día. Dependiendo de una demanda entre un conjunto de ciudades, precio del billete y la opción de usar distintas aeronaves. Además de eso, se especifican las distancia entre las ciudades, la capacidad de cada avión, velocidad, máximo tiempo de utilización y su costo de uso en un periodo de tiempo. A continuación se puede ver gráficamente la distribución de las ciudades y la distancia entre ellas.

\vspace{1em}

\begin{figure}[h]
    \centering
    \begin{tikzpicture}[auto, node distance=3cm, every loop/.style={},
                        thick,main node/.style={circle,draw,font=\sffamily}]
    
      \node[main node] (1) {1};
      \node[main node] (2) [right of=1] {2};
      \node[main node] (3) [above right of=2] {3};
      \node[main node] (4) [below right of=2] {4};
    
      \path[every node/.style={font=\sffamily\small}]
        (1) edge node [above] {260\textsubscript{km}} (2)
        (3) edge node [below right] {375\textsubscript{km}} (2)
        (4) edge node [below left] {310\textsubscript{km}} (2);
    \end{tikzpicture}
    \caption{Grafo de las ciudades}
    \label{fig:cities}
\end{figure}

\paragraph{Demanda}

La demanda de vuelos para cada par de ciudades viene expresada de la siguiente manera. Siendo esta representación útil para resolver el problema en \texttt{Octave} pero sigue teniendo la mismo información que se podría expresar como la figura \ref{fig:cities}.

\begin{table}[h!]
    \centering
    \begin{tabular}{r|r|r|r|r}
        $\lambda_{ij}$%
               &   1  &   2  &   3  &   4\\
            \hline
            \hline
            1  &   0  & 450  &   0  &   0\\
            2  & 600  &   0  & 450  & 760\\
            3  &   0  & 500  &   0  &   0\\
            4  &   0  & 700  &   0  &   0\\
    \end{tabular}
    \caption{Demandas de vuelo}
    \label{tab:demand}
\end{table}

\paragraph{Tasa aérea}

Los costos de los vuelos son iguales independientemente de la dirección en la que se viaje, por lo que el siguiente cuadro es simétrico. A diferencia del anterior (\ref{tab:demand}). Es importante destacar que la unidad de la moneda es indiferente.

\begin{table}[h!]
    \centering
    \begin{tabular}{r|r|r|r|r}
        $f_{ij}$%
               &   1  &   2  &   3  &   4\\
            \hline
            \hline
            1  &   0  & 175  &   0  &   0\\
            2  & 175  &   0  & 230  & 200\\
            3  &   0  & 230  &   0  &   0\\
            4  &   0  & 200  &   0  &   0\\
    \end{tabular}
    \caption{Tasa aérea}
    \label{tab:fare}
\end{table}

\paragraph{Aviones}

En este problema se toman en consideración dos tipos de aeronaves y estas poseen distintas características que se puede ver en el siguiente cuadro. Resaltando 4 aspectos importantes que ayudarán a elaborar las restricciones que ayudarán a conseguir la solución óptima.

\begin{table}[h!]
    \centering
    \begin{tabular}{r|r|r}
        Característica               &    1 &    2\\
        \hline
        \hline
        Capacidad                    &   50 &  100\\
        Velocidad (km/h)             &  400 &  425\\
        Costo de operación (\euro/h) & 1850 & 3800\\
        Uso máximo (h/day)           &   13 &   12\\
    \end{tabular}
    \caption{Tipos de avión}
    \label{tab:planes}
\end{table}

\section{Función objetivo}

Ahora, el primer paso es definir la función que se desea optimizar y decidir si maximizar o minimizar. Para ello, se especifican la variables que definen la solución del problema. En nuestro caso son, número de aviones dado un tipo y el número de vuelos por día desde una ciudad a otro usando cierta aeronave. Formalmente se puede definir de la siguiente manera.

\begin{table}[h!]
    \centering
    \begin{tabular}{r l}
        $P_{k}$   & Número de aviones de tipo $k$\\
        $N_{ijk}$ & Número de vuelos desde $i$ hasta $j$ con el avión $k$\\
         & \\
        $k$ & $= 1,2$\\
        $i,j$ & $= 1,2,3,4$
    \end{tabular}
    \caption{Variables del problema}
    \label{tab:variables}
\end{table}

Como se desea maximizar ganancias, es notorio que el costo del vuelo y la demanda están relacionados, e implican los beneficio de la aerolínea. Usando los cuadros \ref{tab:demand} y \ref{tab:fare}, se puede expresar formalmente la deducción bruta por venta de billetes de la siguiente manera.

\begin{equation}
    \sum_{i,j} \lambda_{ij}.f_{ij}
\end{equation}

Obviamente existe un costo por el uso de las aeronaves y tomando en consideración el número de vuelos de una ciudad a otro usando cierto avión, se puede expresar ese costo con la siguiente ecuación.

\begin{equation}
    \sum_{i,j}\sum_{k} N_{ijk}.C_{ijk}
\end{equation}

Donde $C_{ijk}$ es el costo de volar desde $i$ hasta $j$ usando la aeronave $k$. Los valores de esta matriz de 3 dimensiones se pueden deducir usando la distancia entre dos ciudades, el tiempo que demora cada aeronave (ver tablas \ref{tab:tij1} y \ref{tab:tij2}) y el costo de uso. Dejando el siguiente cuadro como resultado en el caso de la primera aeronave.

\begin{table}[h!]
    \centering
    \begin{tabular}{r|r|r|r|r}
        $C_{ij1}$%
               &   1  &   2  &   3  &   4\\
            \hline
            \hline
              1 &          0 &  1202.5  &          0  &          0\\
              2 & 1202.5 &           0  & 1734.4  & 1433.8\\
              3 &          0 &  1734.4  &          0  &          0\\
              4 &          0 &  1433.8  &          0  &          0\\
    \end{tabular}
    \caption{Costo de volar de i a j con el avión 1}
    \label{tab:cij1}
\end{table}

Y si usamos la segunda aeronave quedaría de la siguiente manera. 

\begin{table}[ht!]
    \centering
    \begin{tabular}{r|r|r|r|r}
        $C_{ij2}$%
               &   1  &   2  &   3  &   4\\
            \hline
            \hline
              1 &      0 &  2324.7 &       0 &       0\\
              2 & 2324.7 &       0 &  3352.9 &  2771.8\\
              3 &      0 &  3352.9 &       0 &       0\\
              4 &      0 &  2771.8 &       0 &       0\\
    \end{tabular}
    \caption{Costo de volar de i a j con el avión 2}
    \label{tab:cij2}
\end{table}

Entonces la función objetivo estaría definida por la ecuaciones descritas anteriormente, con el objetivo de maximizar la ganancia neta.

\begin{equation}
    Max \sum_{i,j} \lambda_{ij}.f_{ij} - \sum_{i,j}\sum_{k} N_{ijk}.C_{ijk}
\end{equation}

O simplemente minimizar los costos de ejecución de la aerolínea, con lo cual quedaría el segundo operando de la resta pero con signo positivo.

\begin{equation}
    Min \sum_{i,j}\sum_{k} N_{ijk}.C_{ijk}
\end{equation}

\section{Restricciones}

Ya que tenemos definida la función objetivo, necesitamos guiar la solución a donde nos interesa, para ello se formalizan las restricciones a las que está sujeto el problema. En primer lugar, nos interesa establecer una relación entre el tiempo de los vuelos y el tiempo que estos están disponibles. Como ya se había mencionado, cada avión tiene un tiempo máximo de uso. En el siguiente apartado se formaliza esta restricción.

\subsection{Máximo uso de las aeronaves}

\begin{equation}
    \sum_{i,j} t_{ijk}.N_{ijk} \leq U_{k}.P_{k},\ \forall k
\end{equation}

Donde $U_k$ es el tiempo de uso máximo de cada avión, este se puede ver en el cuadro \ref{tab:planes}. El tiempo de vuelo entre la ciudad $i$ y $j$ usando la aeronave $k$ se obtiene después de dividir la velocidad de cada avión entre la distancia de la ruta. Los valores obtenidos de esa operación son los siguientes; cada cuadro correspondiente a un tipo de avión.

\newpage

\begin{table}[ht!]
    \centering
    \begin{tabular}{r|r|r|r|r}
        $t_{ij1}$%
               &   1  &   2  &   3  &   4\\
            \hline
            \hline
             1 &       0  & 0.65000  &       0  &       0\\
             2 & 0.65000  &       0  & 0.93750  & 0.77500\\
             3 &       0  & 0.93750  &       0  &       0\\
             4 &       0  & 0.77500  &       0  &       0\\
    \end{tabular}
    \caption{Tiempo de volar de i a j con el avión 1}
    \label{tab:tij1}
\end{table}

\begin{table}[ht!]
    \centering
    \begin{tabular}{r|r|r|r|r}
        $t_{ij2}$%
               &   1  &   2  &   3  &   4\\
            \hline
            \hline
             1 &       0  & 0.61176  &       0  &       0\\
             2 & 0.61176  &       0  & 0.88235  & 0.72941\\
             3 &       0  & 0.88235  &       0  &       0\\
             4 &       0  & 0.72941  &       0  &       0\\
    \end{tabular}
    \caption{Tiempo de volar de i a j con el avión 2}
    \label{tab:tij2}
\end{table}

\subsection{Suplir la demanda}

Una restricción importante, además de maximizar ganancias, es mantener la satisfacción de la clientela. Cubrir la demanda es lo mínimo que se necesita para obtener la mayor ganancia posible, eso para todas las rutas establecidas por la aerolínea. Un factor importante es la capacidad de cada avión, y multiplicando por el número de vuelos, este debe ser mayor o igual a la demanda de una ruta. Formalmente se define esta restricción a continuación.

\begin{equation}
    \sum_{k} n_{k}.N_{ijk} \geq \lambda_{ij},\ \forall (i,j)
\end{equation}

\subsection{Mínimo número de vuelos por ruta}

Para la aerolínea es importante tener un número mínimo de vuelos para cada ruta, este valor es definido de antemano y para nuestro caso se asigno al menos un vuelo por ruta. Es de esperar que se deben sumar los vuelos de ambas aeronaves para definir esta restricción.

\begin{equation}
    \sum_{k} N_{ijk} \geq (N_{ij})_{min},\ \forall (i,j)
\end{equation}

\section{Solución óptima}

Después de implementar un algoritmo para encontrar la solución al problema, donde se usa el método \emph{Simplex} a través del comando \texttt{glpk} de \texttt{Octave}. Se obtuvo un conjunto de soluciones para distintos valores que puede tomar la variable $P_{k}$, estas se pueden adaptar a distintos escenarios en los que se encuentre la aerolinea. A continuación se muestra el cuadro resultante con la ganacia máxima y el número de aeronaves para cada tipo.

\begin{table}[ht!]
    \centering
    \begin{tabular}{r | r | r l}
            Ganancia  &   $P_1$  & $P_2$ & \\
            \vspace{-15pt}\\
            \cline{1-3}
            % \cline{1-3}
            593572  & 0  & 3 & \\
            596153  & 1  & 2 & (a)\\
            596313  & 1  & 3 & (b)\\
            596153  & 2  & 2 & \\
            596313  & 2  & 3 & (c)\\
            594607  & 3  & 1 & \\
            596153  & 3  & 2 & \\
            596313  & 3  & 3 & (d)\\
    \end{tabular}
    \caption{Ganancia para distintos valores de $P_k$}
    \label{tab:ganancia}
\end{table}

Se quieren resaltar dos de los resultados obtenidos. El primero es (b), se puede ver que comparte la misma ganacia que los casos (c) y (d), pero posiblemente la aerolínea tenga interés en reducir el número de aeronaves por lo que la opción (b) sería la mejor. Aunque, es posible que la disponibilidad de aeronaves puede ser un incoveniente y se opte por el caso (a), reduciendo el número de aeronaves. A continuación se muestran la mejor distribución de vuelos encontrada por el algoritmo en el caso (b) para cada tipo de avión.

\begin{table}[ht!]
    \centering
    \begin{tabular}{r|r|r|r|r}
        $N_{ij1}$%
               &   1  &   2  &   3  &   4\\
            \hline
            \hline
             1 & 0 & 1 & 0 & 0\\
             2 & 0 & 0 & 1 & 0\\
             3 & 0 & 0 & 0 & 0\\
             4 & 0 & 0 & 0 & 0\\
    \end{tabular}
    \caption{Mejor distribución de vuelos para el avión 1 (b)}
    \label{tab:nij1b}
\end{table}

\begin{table}[ht!]
    \centering
    \begin{tabular}{r|r|r|r|r}
        $N_{ij2}$%
               &   1  &   2  &   3  &   4\\
            \hline
            \hline
             1 & 0 & 4 & 0 & 0\\
             2 & 6 & 0 & 4 & 8\\
             3 & 0 & 5 & 0 & 0\\
             4 & 0 & 7 & 0 & 0\\
    \end{tabular}
    \caption{Mejor distribución de vuelos para el avión 2 (b)}
    \label{tab:nij2b}
\end{table}

En el caso de (a), los valores de $N_{ijk}$ quedarían de la siguiente manera.

\begin{table}[ht!]
    \centering
    \begin{tabular}{r|r|r|r|r}
        $N_{ij1}$%
               &   1  &   2  &   3  &   4\\
            \hline
            \hline
             1 & 0 & 5 & 0 & 0\\
             2 & 0 & 0 & 1 & 0\\
             3 & 0 & 0 & 0 & 0\\
             4 & 0 & 0 & 0 & 0\\
    \end{tabular}
    \caption{Mejor distribución de vuelos para el avión 1 (a)}
    \label{tab:nij1a}
\end{table}

\begin{table}[ht!]
    \centering
    \begin{tabular}{r|r|r|r|r}
        $N_{ij2}$%
               &   1  &   2  &   3  &   4\\
            \hline
            \hline
             1 & 0 & 2 & 0 & 0\\
             2 & 6 & 0 & 4 & 8\\
             3 & 0 & 5 & 0 & 0\\
             4 & 0 & 7 & 0 & 0\\
    \end{tabular}
    \caption{Mejor distribución de vuelos para el avión 2 (a)}
    \label{tab:nij2a}
\end{table}

\section{Conclusiones}

Después de elaborar una solución a través de la programación lineal se puede ver el poder que tienen estos métodos para resolver problemas de optimización. Muchos de estos problemas representan escenarios de la vida real, logrando usar un programa matemático en distintos tipos de empresas para mejorar diferentes aspectos que se puedan modelar. En concreto, el problema de la programación de vuelos puede presentar variantes pero en este trabajo se pudo establecer un criterio para una solución óptima y al final sigue dependiendo factores ajenos al modelo implementado. Es importante dar una interpretación adecuada a los resultados, o restringir aun más el espacio de solución para ajustar el modelo.

El uso de herramientas adecuadas también es importante porque permite elaborar una solución de manera sencilla, ya que se ha desarrollado para ello; computación numérica. Brinda la posibilidad de enfocar exclusivamente a manejar los datos y ofrece funcionalidades que, sabiendo usarlas de manera inteligente, pueden aplicarse distintos problemas. No solo hay que saber usar las herramientas, sino la teoría que las ha fundamentado.hay que saber usar las herramientas, sino la teoría que las ha fundamentado.

\end{document}